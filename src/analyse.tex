%!TEX encoding = UTF-8 Unicode

\chapter{Analyse}

\section{Thèse}

Plusieurs aspects valides de l'électrification des transports indiquent que cette technologie ne s'inscrit pas dans le mouvement du développement durable.
Plusieurs points tels que des points socioéconomiques et environnementaux appuient cet énoncé.

\subsection{Aspects socioéconomiques}

Plusieurs perspectives socioéconomiques nous indiquent que cette technologie ne s'inscrit pas dans le mouvement du développement durable. Par exemple, les subventions gouvernementales sont un aspect qui peut constituer une injustice sociale. En effet, les subventions gouvernementales sont dédiées majoritairement aux acheteurs de véhicules électriques individuels. Toutefois, les automobiles électriques individuels sont plus chères que les véhicules fonctionnant aux carburants fossiles. Ainsi, seules les personnes ayant déjà, à priori, des bons revenus peuvent se les payer\wcite{Pineau2016}\wcite{Durand2014}. Or, les subventions ont pour but d'aider à financer l'achat de ces automobiles pour les consommateurs à revenus moyens. Cependant, comme seules les gens ayant déjà des bons revenus peuvent se payer ces véhicules, seulement ces derniers recevront les subventions. En d'autres mots, le gouvernement favorise les mieux nantis avec ces subventions.

Par ailleurs, certains gouvernements favorisent l'achat d'automobiles électriques au détriment de ceux fonctionnant à l'énergie fossile pour les transports en commun\wcite{Delorme2012}. Toutefois, ces coûts plus élevés aurait pu être investis dans des projets au bénéfice des contribuables.

\subsection{Aspects environnementaux}

Aussi, des aspects environnementaux appuient l'opinion que les véhicules électriques ne s'inscrivent pas dans le mouvement du développement durable. L'augmentation d'automobiles électriques sur les routes implique une augmentation de consommation d'énergie électrique. Par conséquent, les installations de production doivent produire plus d'électricité pour subvenir à cette hausse de la demande. Toutefois, ce ne sont pas toutes les centrales qui utilisent des énergies renouvelables pour produire leur électricité. Par exemple, la Chine utilise le charbon comme combustible dans plusieurs de ses centrales\wcite{Zhang2010}. Alors, pour les pays qui utilisent des sources d'énergie non renouvelables, l'avantage que les voitures électriques ont par rapport à leur empreinte écologique est contrebalancé par l'empreinte des centrales. 

Par ailleurs, les batteries des voitures électriques produisent aussi des déchets qui sont non négligeables\wcite{Durand2014}. En effet, la production de ces batteries est bien souvent faite hors du pays où le véhicule a été fabriqué. Cela implique donc une certaine empreinte écologique pour le transport de ces batteries.

\section{Antithèse}

Bien que certains aspects de l'électrification des transports pourraient indiquer que cette technologie ne semble pas s'inscrire dans le mouvement du développement durable, d'autres perspectives pourraient nous indiquer le contraire. 

\subsection{Aspects socioéconomiques}

Bien que les automobiles électriques soient plus dispendieuses que leurs homologues à carburants fossiles, ceux-ci permettent d'économiser sur l'achat des carburants sur de grandes distances\wcite{Durand2014}. Ainsi, elles pourraient être avantageuses d'un point de vue économique pour les compagnies de transport.

De plus, il y a la possibilité de créer des emplois avec l'industrie de l'électrification des transports. Par exemple, au Québec, \shortquote{Le gouvernement a annoncé lors du dévoilement de sa politique énergétique que 5000 emplois seraient créés dans ce secteur [\textellipsis]}{Breton2016}.

\subsection{Aspects environnementaux}

Par ailleurs, les automobiles électriques diminuent notre dépendance aux carburants fossiles. La majorité de la consommation de ces derniers provient du secteur des transports. Par exemple, au Québec, la consommation du pétrole par cette industrie correspond à 70\% de la consommation totale\wcite{Delorme2012}. Par conséquent, une diminution de cet usage entraîne une réduction des émissions de gaz à effets de serre. En effet, \shortquote{[\textellipsis] plus de 50\% des émissions québécoises sont directement attribuables à l'utilisation du pétrole}{Delorme2012}.

\section{Synthèse}

L'électrification des transports ne s'inscrit pas complètement dans le mouvement du développement durable. Pour que l'électrification des transports s'inscrive dans ce mouvement, il faudrait que cela aide à diminuer les émissions de gaz à effets de serre et la dépendance au pétrole. Toutefois, on remarque que ce n'est pas nécessairement le cas pour les voitures électriques individuels.

En effet, on pourrait dire que seuls les transports en commun s'inscrivent dans ce mouvement. L'utilisation de voitures électriques par les particuliers ne diminue pas la congestion routière et c'est ce problème qui cause majoritairement l'émission des gaz à effets de serre\wcite{Pineau2016}. Ainsi, une des solutions ne serait pas d'utiliser des automobiles électriques, mais plutôt de changer la mentalité des gens afin qu'ils priorisent les transports en commun, qui, eux, pourraient être électrifiés. 

Par ailleurs, l'utilisation d'électricité comme énergie n'implique pas que la source d'électricité est propre. Donc, il faudrait changer les sources d'énergie par quelque chose de durable afin de ne pas perdre tous les bénéfices de l'électrification. 

Finalement, la vraie solution serait de combiner toutes ces réponses dans une seule. Il faudrait donc donner des incitatifs aux gens afin de les pousser vers le transport en commun, qui, lui, serait électrifié et dont la source d'énergie serait propre. Aussi, les batteries de ces véhicules devraient être produites localement afin d'éviter la taxe écologique liée au transport de ces dernières.
