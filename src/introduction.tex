%!TEX encoding = UTF-8 Unicode

\chapter{Introduction}

La domestication du cheval marque une époque importante dans l'histoire de l'homme. En effet, avec le cheval, l'homme put se déplacer qu'il ne le pouvait avant. Avec l'industrialisation, l'homme délaissa le cheval et adopta un nouveau mode de transportation: la voiture. L'homme inventa également d'autres moyens de transport tel que le train, l'avion, etc. Toutefois, au milieu du $21^{\mathrm{e}}$ siècles, le mouvement du développement durable commença à émerger. Ce mouvement déclarait que l'humain devrait avoir un meilleur rapport avec la nature. De nos jours, avec le réchauffement climatique, ce mouvement gagne de plus en plus en popularité et l'homme se met à essayer de développer de nouveaux moyens de transport pour remplacer, à l'instar du cheval, les automobiles à essence par des automobiles fonctionnant à l'électricité. Toutefois, bien que beaucoup de gens considère ces nouveaux automobiles comme le futur du transport, il est possible de se questionner si ces automobiles adhèrent bel et bien au mouvement du développement durable.
