%!TEX encoding = UTF-8 Unicode

\chapter{Introduction}

Depuis le début de son histoire, l'humanité a connu de nombreux progrès techniques au niveau de ses déplacements. De la marche au cheval et du cheval à la voiture, ses méthodes de déplacement ont fortement évoluées. Toutefois, l'avénement de la voiture a apporté une nouvelle problématique jusqu'à ce moment inexistante: la pollution. Par contre, la voiture n'est pas le seul véhicule ayant cette problématique. Les véhicules tel que l'avion, le bateau, l'hélicoptère, etc. ont aussi le même problème. Avec l'adoption de ces nouveaux moyens de transport et avec l'apparition de l'industrialisation, un mouvement connu sous le nom de \og{}développement durable\fg{} est apparu. Ce mouvement essaie de minimiser l'impact écologique que l'humain a sur son écosystème. De nombreux adeptes de ce mouvement se questionnent quant à savoir si l'électrification de ces transports automobiles pourrait être une solution qui correspondrait à ce mouvement. Certains disent que l'électrification des transports automobiles ne s'inscrit pas dans le mouvement du développement durable, alors que d'autres personnes disent que, oui, elle s'y inscrit. Ce document est une analyse pour répondre à la question duquel des deux camps a raison. Pour ce faire, une démarche critique sera adoptée par rapport aux sources choisies dans le but d'avoir une opinion objective du contexte. Puis, avec cette vision objective, une analyse de la question sera effectuée.
