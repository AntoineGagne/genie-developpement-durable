%!TEX encoding = UTF-8 Unicode

\begin{filecontents}{bibliographie.bib}
@online{foo12,
    year  = {2012},
    title = {footnote-reference-using-european-system},
    url   = {http://tex.stackexchange.com/questions/69716/footnote-reference-using-european-system},
}

@article{test,
    author   = {Test T.},
    title    = {Synthesis of Enantiopure Alcohols},
    volume   = {71},
    number   = {17},
    journal  = {J. Org. Chem.},
    month    = {aug},
    year     = {2006}
}
@book{Poincare:1968-ORIG,
    author    = {Jules-Henri Poincaré},
    title     = {La science et l’hypothèse},
    publisher = {Flammarion},
    location  = {Paris},
    date      = {1968},
    related   = {Poincare:1968-ITA}
}
@book{schramm,
    author    = {Percy Ernst Schramm},
    title     = {Hamburg, Deutschland und die Welt},
    subtitle  = {Leistungen und Grenzen hanseatischen Bürgertums in der Zeit zwischen Napoleon I. und Bismarck},
    edition   = {2},
    location  = {Hamburg},
    publisher = {Hoffmann \& Campe},
    date      = {1952},
}
@book{hanham,
    author    = {Harold John Hanham},
    title     = {Elections and Party Management},
    subtitle  = {Politics in the Age of Disraeli and Gladstone},
    location  = {London},
    date      = {1959},
    publisher = {Longman},
}
@book{BlackEley,
    author    = {David Blackbourn and Geoff Eley},
    title     = {Mythen deutscher Geschichtsschreibung},
    subtitle  = {Die gescheiterte bürgerliche Revolution von 1848},
    location  = {Frankfurt am Main and Berlin and Wien},
    date      = {1980},
    publisher = {Ullstein},
}
@book{Lutz,
    author    = {Heinrich Lutz},
    title     = {Reformation und Gegenreformation},
    location  = {Frankfurt am Main and Berlin and Wien},
    date      = {1982},
    publisher = {Oldenbourg Wissenschaftsverlag},
    edition   = {2},
    location  = {München and Wien},
    series    = {Oldenbourg Grundriß der Geschichte},
    number    = {10},
    pagetotal = {251},
}
@article{zorn,
    author    = {Wolfgang Zorn},
    title     = {Wirtschafts- und sozialgeschichtliche Zusammenhänge der deutschen Reichsgründung (1850–1879)},
    journal   = {HZ},
    volume    = {197},
    date      = {1963},
    pages     = {318–34},
}
@collection{schieder,
    editor    = {Theodor Schieder},
    title     = {Beiträge zur britischen Geschichte im 20. Jahrhundert},
    location  = {München},
    date      = {1983},
    series    = {Historische Zeitschrift, Beihefte},
    number    = {8},
    edition   = {1},
}
@incollection{alter,
    author    = {Peter Alter},
    title     = {Der britische Generalstreik von 1926 als politische Wende},
    pages     = {89–116},
    crossref  = {schieder},
}
@phdthesis{lacher,
    author   = {Hugo Lacher},
    title    = {Politischer Katholizismus und kleindeutsche Reichsgründung},
    subtitle = {Eine Studie zur politischen Ideenwelt im deutschen Katholizismus},
    location = {Mainz},
    date     = {1963},
}
\end{filecontents}
