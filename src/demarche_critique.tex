%!TEX encoding = UTF-8 Unicode

\chapter{Démarche critique}

L'article \quotify{Le Québec mise sur l’électrification des transports} d'Emily Laperrière contient très peu d'informations techniques sur le sujet. Toutefois, sa valeur est plus sur les idées que le texte apporte. En effet, certains des experts consultés dans l'article soulève des points comme le fait que l'électrification des véhicules n'est pas une solution définitive. 

Le texte \quotify{L’électrification des transports: une perspective québécoise} d'Audrey Durand, Nicolas Lavigne-Lefebvre et Jean-François Rougès est intéressant autant dans son contenu que dans sa manière de le présenter. En effet, le texte est assez neutre et ne se contente pas de lister les différents avantages associés à l'électrification des véhicules. Il aborde également des enjeux sociaux tel que le changement de paradigme nécessaire pour que la solution des véhicules électriques soit vraiment efficace. Il tient aussi compte de l'impact économique des véhicules électriques, autant au niveau des particuliers que des entreprises et gouvernements. Il mentionne aussi les impacts écologiques des véhicules électriques et considèrent aussi les aspects négatifs des voitures électriques par rapport au plan écologique. Les auteurs sont assez crédibles puisqu'il regroupe différents organismes spécialisés dans les domaines qu'aborde le texte.

Le texte \quotify{\emph{Environmental Implication of Electric Vehicles in China}} de Hong Huo, Qiang Zhang, Michael Q. Wang, David G. Streets et Kebin He présente des aspects intéressants, mais contient certaines lacunes. En effet, le texte aborde seulement les véhicules électriques en Chine. Ainsi, les points qu'il soulève concernent principalement la Chine. Toutefois, il est quand même possible d'en tirer des informations intéressantes puisque les circonstances dont il parle peuvent se retrouver dans d'autres pays. Également, des points qui sont soulevés dans le texte sont aussi applicables dans d'autres pays. Par exemple, le texte mentionne le fait que les véhicules ne sont peut-être pas une solution écologique si excellente en raison de la source majoritaire avec laquelle la Chine produit son électricité. On peut donc reprendre cette critique pour les pays qui, eux-aussi, n'ont pas une source durable d'électricité.

Le texte \quotify{Électrification des transports: des onjectifs ambitieux... et une cooupe de 80\%} est rédigé par M. Daniel Breton.M. Daniel Breton est un blogueur et consultant en matière d’énergie, d’environnement et d’électrification des transports. Le texte est subjectif, puisqu’il est consulté sur son blogue. Par contre, l’auteur possède de l’expérience comme ministre du Développement durable et d’en d’autres domaines connexes, donc nous le considérons crédible. Cette source fournie plusieurs données socioéconomiques par rapport aux investissements du gouvernement québécois sur le plan d’électrification des transports et sur les annonces gouvernementales reliées à la création d’emploi dans le milieu. En résumé, le texte énonce seulement l’évidence concernant l’électrification des transports en rajoutant quelques chiffres sur les annonces gouvernementales. La principale limite de la source, c’est qu’elle ne rentre pas suffisamment dans le cœur du sujet. L’auteur décrit seulement les éléments socioéconomiques entourant l’électrification des transports. Les sources complémentaires doivent davantage élaborer l’aspect environnemental et approfondir les impacts sociaux et économiques.  

Le texte \quotify{Électrification des transports au Québec: du mythe à la réalité - à quelle vitesse? } est rédigé par M.Pierre Delorme.
M. Pierre Delorme est un professeur au Département d’études urbaines et touristique de l’université du Québec à Montréal. Le texte s’appuie principalement sur des données statistiques pour appuyer ses arguments, donc la source est crédible. L’auteur explique les avantages de l’électrifications des transports vis-à-vis la diminution de consommations du pétrole et sur la diminution des émissions des gaz à effet de serres au Québec en faisant des comparaisons avec d’autres villes. Il compare les coûts reliés à l’énergie fossile et à l’énergie électrique tout en comparant les moyens de transports individuels et communs. Bref, la source est complète et décrit bien le contexte québécois de l’électrification des transports. Par contre, le texte décrit principalement la situation au Québec. Il est donc difficile d’affirmer si cette solution est durable et généralisable pour l’ensemble des pays. 

La vidéo de  \quotify{La Maison du développement durablee} est très intéressante puisqu’elle présente le point de vue de divers intervenants dans le domaine de l’électrification des transports. Le premier panéliste, M. Pierre Olivier Pineau, considère que l’électrification des transports ne devrait pas être une priorité puisqu’elle n’est pas une solution au problème. De plus, il critique le système de subvention québécois, puisque ce sont davantage le riche qui en profite et que le budget n’est pas totalement dépensé. Il propose même une piste de solution pour régler le problème. Dans le même ordre d’idée, le second panéliste, M. Sylvain Ouellet, explique que les véhicules électriques ne répondent pas aux enjeux environnementaux et qu’elle est extrêmement couteuse et inéquitable socialement à cause des subventions. En résumé, la source est complète, car elle décrit les impacts environnementaux, sociaux et économiques reliés à l’électrification des transports. Par contre, elle explique seulement les enjeux pour le contexte québécois, donc elle n’est pas universelle.  


