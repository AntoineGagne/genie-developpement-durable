%!TEX encoding = UTF-8 Unicode

\chapter{Démarche critique}

L'article \og{} Le Québec mise sur l’électrification des transports\fg{} d'Emily Laperrière contient très peu d'informations techniques sur le sujet. Toutefois, sa valeur est plus sur les idées que le texte apporte. En effet, certains des experts consultés dans l'article soulève des points comme le fait que l'électrification des véhicules n'est pas une solution définitive. 

Le texte \og{} L’électrification des transports: une perspective québécoise\fg{} d'Audrey Durand, Nicolas Lavigne-Lefebvre et Jean-François Rougès est intéressant autant dans son contenu que dans sa manière de le présenter. En effet, le texte est assez neutre et ne se contente pas de lister les différents avantages associés à l'électrification des véhicules. Il aborde également des enjeux sociaux tel que le changement de paradigme nécessaire pour que la solution des véhicules électriques soit vraiment efficace. Il tient aussi compte de l'impact économique des véhicules électriques, autant au niveau des particuliers que des entreprises et gouvernements. Il mentionne aussi les impacts écologiques des véhicules électriques et considèrent aussi les aspects négatifs des voitures électriques par rapport au plan écologique. Les auteurs sont assez crédibles puisqu'il regroupe différents organismes spécialisés dans les domaines qu'aborde le texte.

Le texte \og\emph{Environmental Implication of Electric Vehicles in China}\fg{} de Hong Huo, Qiang Zhang, Michael Q. Wang, David G. Streets et Kebin He présente des aspects intéressants, mais contient certaines lacunes. En effet, le texte aborde seulement les véhicules électriques en Chine. Ainsi, les points qu'il soulève concernent principalement la Chine. Toutefois, il est quand même possible d'en tirer des informations intéressantes puisque les circonstances dont il parle peuvent se retrouver dans d'autres pays. Également, des points qui sont soulevés dans le texte sont aussi applicables dans d'autres pays. Par exemple, le texte mentionne le fait que les véhicules ne sont peut-être pas une solution écologique si excellent en raison de la source majoritaire avec laquelle la Chine produit son électricité. On peut donc reprendre cette critique pour les pays qui, eux-aussi, n'ont pas une source durable d'électricité.
