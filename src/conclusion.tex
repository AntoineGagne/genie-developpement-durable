%!TEX encoding = UTF-8 Unicode

\chapter{Conclusion}

De l'analyse faite précedemment, nous avons pu remarqué qu'il existait deux points quant à savoir si l'électrification des transports s'inscrit dans le mouvement du développement durable ou non. 

Ainsi, le premier camp stipulait que les véhicules électriques permettent de diminuer la dépendance aux énergies fossiles tout en diminuant les gaz à effets de serre.
Il disait aussi que cette technologie avait des opportunités de créer des emplois dans certains cas.

L'autre camp, quant à lui, soulevait des points qui venaient à dire que l'électrification des transports n'est pas une solution viable. Selon eux, l'électricité produite par les centrales à l'aide de moyens non renouvelables vient contrebalancer les bénéfices que pourraient avoir les véhicules électriques. Par ailleurs, les batteries produites à l'étranger contribuaient aussi à l'empreinte écologique de par le transport requis pour les acheminer vers les usines de fabrication. De plus, certains gouvernements accordent des subventions aux acheteurs de véhicules électriques qui favorisent des personnes qui n'ont pas réellement besoin de ces subventions au lieu d'aider ceux qui sont visés par celle-ci.

En prenant des points des deux camps, il a été possible de conclure que l'électrification des véhicules fait partiellement partie du développement durable, mais que ce n'est pas en solution en soi. Il a été soulevé que le trafic était une des causes majeures de la consommation de pétrole et de l'émission de gaz à effets de serre des voitures standards. Pour cette raison, la solution retenue impliquerait un changement de paradigme favorisant le transport en commun qui, lui, bénéficierait de l'électrification. Toutefois, ce changement de paradigme n'est pas complet en soi et devrait s'accompagner de changements de production d'énergie pour des solutions plus durables.
